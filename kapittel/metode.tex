Metode handler om hvordan man innhenter informasjon, og hvordan data skal analyseres og tolkes. Videre er det vanlig å dele metodelære inn i to hovedmetoder;
\begin{itemize}
\item\textit{Kvantitativ metode} - Befatter seg med tall og det som er målbart (kvantifiserbart) \cite{kvantitativMetode}. Metoden gir en høy grad av generaliserbarhet basert på tallfestet data. En rask og oversiktlig metode for å innhente kvantitativ data på er gjennom spørreundersøkelser. En svakhet ved spørreundersøkelser er imidlertid faren ved å overse informasjon som ikke kan tallfestes. Dette kan føre til at informasjonen gir et overfladisk og forenklet bilde av virkeligheten.
\item\textit{Kvalitativ metode} - Generering av kunnskap basert på menneskelig erfaring og fortolkning, og brukes ofte til å undersøke sosiale prosesser og sosialt samspill  \cite{kvalitativMetode}. En av de vanligste teknikkene for innhenting av kvalitative data er ved intervjuer. Teknikken benyttes for å sikre en dypere forståelse for emnet man ønsker å undersøke.
\end{itemize}

I oppgaven har vi benyttet begge metodene for å generalisere funnene og samtidig få en dypere forståelse av årsakene som ligger bak. 

\section{Datainnsamling}
Prosessen i datainnsamlingen startet med uformelle samtaler med tidligere daglig leder, nå kommunikasjonsrådgiver, Kjetil Væhle. Ett av gruppemedlemmene kjente Væhle fra før, og det var derfor naturlig å benytte denne relasjonen til å få grunnleggende informasjon om virksomheten og markedet den opererer i. Videre fikk vi tidlig i prosessen tilgang til sekundærdata i form av en medarbeiderundersøkelse som hadde blitt gjennomført i januar 2019. Undersøkelsen ga inntrykk av utfordringer knyttet til ledelsen. På bakgrunn av undersøkelsen og de innledende samtalene fikk vi utarbeidet den aktuelle problemstillingen, og innhentet videre primærdata gjennom en spørreundersøkelse og to dybdeintervjuer. I tillegg har vi gjennom hele arbeidsprosessen hatt mulighet til å kontakte Væhle ved behov.

\section{Spørreundersøkelse}
Spørreundersøkelsen ble gjennomført av 15 ansatte via web (Qualtrics). Spørreskjemaet bestod av 8 spørsmål med 5 tilhørende svaralternativer, hvor spørsmålene var formulert som påstander. Respondentene kunne velge en score fra 1 (svært uenig) til 5 (svært enig), basert på deres oppfatning om den aktuelle påstanden. 

\section{Dybdeintervjuer}
Dybdeintervjuene ble gjennomført på bakgrunn av spørreundersøkelsen, for å få en dypere innsikt og forståelse for resultatene. Vi benyttet en semistrukturert intervjuform hvor samtalen mellom forskeren og intervjuobjektet blir styrt av forskeren. Intervjuobjektene bestod av en medarbeider som jobber i Advertising (ønsket å være anonym) og tidligere daglig leder Kjetil Væhle. I begge tilfeller åpnet vi intervjuene med å be objektene fortelle og dele generelle tanker og oppfatninger av dagens ledelse. Tanken bak dette var å skape en åpen samtale og anledning for deltakerne til å snakke fritt om emnet, uten påvirkning fra spørsmålsstillerne. Utover i intervjuene stilte vi spørsmål knyttet til resultatene fra spørreundersøkelsen og emnene vi ønsket å få en dypere innsikt i. Begge intervjuene ble foretatt separat i Involve sine lokaler.

\section{Refleksjoner rundt validitet, pålitelighet og generaliserbarhet}
Validitet handler om gyldighet, og om man på bakgrunn av resultatene av en studie kan trekke gyldige slutninger om det man ønsker å undersøke \cite{validitet}. For å sikre en høy grad av validitet har vi innhentet data gjennom kvantitative - og kvalitative undersøkelser. Det kan imidlertid stilles spørsmålstegn ved generaliserbarheten, da kun 15 av 32 mulige respondenter besvarte den kvantitative spørreundersøkelsen. I forhold til pålitelighet har vi bevisst valgt å innhente primærdata fra begge parter. I tillegg har kontaktpersonen vår i Involve vært en god kilde til datainnsamling i forhold til å se problemstillingen fra både ledernes- og medarbeidernes perspektiv. 