I dette kapittelet vil vi kort redegjøre for teoretiske modeller og rammeverk som vil bli benyttet i oppgaven. Med utgangspunkt i disse teoriene vil vi analysere ledelsen til Involve, og se hva ledelsen kan gjøre annerledes for å skape bedre forutsetninger for å nå virksomhetens mål.

\section{Koorienteringsmodellen}
Modellen til Mcleod og Chaffee (1973) har til hensikt å forklare hvordan kommunikasjonsproblemer kan oppstå som følge av ulik tolkning og oppfatning blant en leder og interessenter. Modellen bruker fire assosiasjonstyper som definerer og påvirker interaksjonen mellom partene; forståelse, enighet, samsvar og nøyaktighet (Brønn og Arnulf s 157). I følge Dozier og Ehling (1992) er det fire koorienteringstilstander partene kan befinne seg i, hvor de to sistnevnte skaper konflikter;

\begin{itemize}
\item\textit{Sann konsensus} - Partene er enige om å være enige
\item\textit{Dissens} - Partene er enige om å være uenige
\item\textit{Falsk konsensus} - Partene er uenige, men tror de er enige
\item\textit{Falsk konflikt} - Partene er enige, men tror de er uenige
\end{itemize}

Modellen illustrerer hvor lett misforståelser kan føre til konflikter i situasjoner hvor lederen ikke er oppmerksom på om deres oppfatning av de ansattes synspunkter er nøyaktig (hele biten ref fra s 158). I oppgaven har vi benyttet modellen med et formål om å analysere situasjoner i virksomheten som hindrer effektiv ledelse og kommunikasjon.

\section{Makt, styring og ledelse}
Ledelse blir ofte blandet sammen med begreper som makt, styring og autoritet. For å kunne gjennomføre nøyaktige analyser angående Involve sin ledelse, finner vi det hensiktsmessig å definere begrepene, og hvordan ledelse skiller seg fra tradisjonell styring og makt. Makt handler om å påtvinge andre sin egen vilje, og tvinge folk til en bestemt livsførsel. Styring tar for seg selve kjernen i organisering, det vil si regler, prosedyrer, rutiner og systemer. Autoritet er knyttet til personer og betegner enkeltpersoners rett til å utøve styring. I motsetning til disse begrepene, dreier ledelse seg om å skape oppslutning fra folk som prinsipielt kunne villet noe annet, og å mobilisere innsatsvilje og samarbeid mot et felles mål (Brønn og Arnulf s 130)

\indent \newline
Med utgangspunkt i ovennevnte definisjon av ledelse, vil neste avsnitt presentere ledelsesteorier som legges til grunn for å besvare problemstillingen. 

\section{Transformasjonsledelse og transaksjonsledelse}
Transformasjonsledelse er moderne ledelsesteorier som har vokst frem i løpet av de siste 40 årene, og bygger på Max Webers gamle begrep karisma (Brønn og Arnulf s 137). Karisma handler om lederens evne til å inspirere, motivere og appellere til hvert enkelt individs følelsesmessige behov. Det er ansett som en viktig attributt for å lykkes med transformasjonsledelse. Ledere som oppfattes som karismatiske, vil bygge tillit hos de ansatte og et ønske om å ville identifisere seg med lederen (perspektiver på ledelse, s 75).

\indent \newline
Transformasjonsledere evner å utvide og stimulere ansattes interesser gjennom å skape tilhørighet til virksomhetens konkrete og overordnede mål. Samtidig påvirkes ansatte til å se utover sine egne interesser (Perspektiver på ledelse, s 111). Det finnes flere ulike teorier om transformasjonsledelse, men vi har valgt å fokusere på Bernard Bass sin tolkning av teorien. Bass forklarer teorien gjennom fire dimensjoner som omhandler transformasjonslederens karisma. Disse er;

\begin{itemize}
\item\textit{Idealisert innflytelse} - Omhandler lederens evne til å kommunisere verdier og visjoner ved å opptre som en inspirerende rollemodell. Lederen evner å skape innflytelse hos de ansatte gjennom sine handlinger.
\item\textit{Inspirerende motivasjon} - Lederen inspirerer og motiverer medarbeiderne gjennom å skape tilhørighet til felles mål og delte visjoner. Lederen snakker om fremtiden med tiltro til mestring.
\item\textit{Intellektuell stimulering} - Lederen fremmer kreativitet og innovativ atferd blant medarbeiderne. De ansatte føler seg oppmuntret til å arbeide selvstendig med oppgavene og finne nye kreative løsninger på problemer.
\item\textit{Individuelle hensyn} - Medarbeidernes behov for måloppnåelse og vekst blir hensyntatt, og lederen bidrar med støttende, læringsorientert og personlig samhandling.
\end{itemize}
(Punktene er hentet fra https://www.bi.no/forskning/business-review/articles/2013/11/den-inspirerende-leder/)

\indent \newline
Til sammenligning med transformasjonsledelse, der ansattes inspirasjon og motivasjon skal komme innenfra, handler transaksjonsledelse om inspirasjon fra ytre faktorer. For arbeidstakeren oppfattes arbeidsforholdet som en transaksjon der man tilbyr arbeidskraft i bytte mot lønn. Arbeidsforholdet kjennetegnes også av mindre personlige relasjoner mellom leder og medarbeiderne.

\indent \newline
De to prinsippene som best beskriver transaksjonsledelse er \textit{betinget belønning} og \textit{ledelse ved unntak}. Betinget belønning går ut på at arbeidstaker blir lovet belønning for de resultatene de oppnår. Ledelse ved unntak skiller mellom \textit{aktiv} og \textit{passiv} ledelsesstil. Lederen kan gå aktivt inn og lete etter avvik på rutiner og prosedyrer og gripe inn, eller han/hun kan velge å passivt følge med og kun si fra dersom regler ikke blir fulgt. 

\indent \newline
Begge ledelsesteoriene kan danne grunnlag for suksess, men det er transformasjonsledelse som vil kunne danne grunnlag for motiverte medarbeidere som jobber mot et felles overordnet mål. Teoriene er ikke gjensidig utelukkende, og lederen kan kjennetegnes med egenskaper fra begge ledelsesstilene. 

